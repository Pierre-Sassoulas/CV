\documentclass[11pt,a4paper]{moderncv}
\moderncvtheme[green]{classic}
% CV theme - options include: 'casual' (default), 'classic', 'oldstyle' and
% 'banking'
% CV color - options include: 'blue' (default), 'orange', 'green', 'red',
% 'purple', 'grey' and 'black'
\usepackage[top=1cm, bottom=1cm, left=1cm, right=1cm]{geometry}
\usepackage[french]{babel}
\usepackage[utf8x]{inputenc}
\usepackage[T1]{fontenc}
\usepackage{multibib}
\usepackage{pstricks}
\usepackage{fontawesome}
\usepackage{multido}
\SpecialCoor
\makeatletter
\def\LoadPSVars{\pstVerb{/ptcm {\pst@number\psunit div} bind def}}
\makeatother
\def\points{}
\def\Star{%
    \xdef\points{}% cleaning
    \multido{\iR=0+72,\ir=36+72}{5}{\xdef\points{\points (10pt;\iR)(5pt;\ir)}}
    \expandafter\pspolygon\points}
\def\Rating#1{% #1: percentage
    \psscalebox{0.35}{%
    \begin{pspicture}(11pt,-11pt)(25pt,11pt)
    \LoadPSVars
    \psclip{\pscustom{\translate(20pt,0)\Star}}
        \psframe*[linecolor=yellow](11pt,-11pt)(!#1 11 add ptcm 11 ptcm)
    \endpsclip
    \pscustom{\translate(20pt,0)\Star}
    \end{pspicture}}}
\firstname{
  Pierre
}
\familyname{
  Sassoulas
}
\title{
  Ingénieur informatique \newline{}
  Architecture des Systèmes d'Information
}
\email{
  pierre.sassoulas at gmail.com
}
\social[github]{pierre-sassoulas}
\address{XX XXX XXXXXXX XXXXXXXXXX}{XXXXX XXXXXXXXXXXXXXXXXXXXXXXX}

\mobile{00 00 00 00 00}
% \phone{}
\extrainfo{
  Titulaire du permis B
} % \\ et du CACES nacelle}
% CACES Valide uniquement entre 2008 et 2013 (Valable 5 ans)
%\photo[92pt][0.4pt]{picture.jpg}
\makeatletter
\renewcommand*{\bibliographyitemlabel}{\@biblabel{\arabic{enumiv}}}
\makeatother
\newcites{book,misc}{{Books},{Others}}
\nopagenumbers{}

\begin{document}
  \maketitle
  \vspace*{-5mm}
    \section{FORMATION}
    \cventry
    {2009 - 2015}
    {Élève-ingénieur}
    {INSA de Rouen}
    {Architecture des Systèmes d'Information (ASI)}
    {}{}

%  \cventry
%    {2009}
%    {BAC S mention BIEN}
%    {Lycée Alain Borne}
%    {Option biologie}
%    {Montélimar}
%    {}

%  \cventry
%    {2008}
%    {Caces Nacelle}
%    {CEFTIC de Pierrelatte}
%    {}{}{}

  \section{EXPÉRIENCE PROFESSIONNELLE}
  \cventry
    {\textbf{python}, \textbf{docker}, \textbf{C++}}
    {Consultant technique}
    {Antidot}
    {Lyon}
    {Septembre 2018 à ce jour}{
      Développement de modules spécifiques ; support de niveau 3 et gestion d'un pipeline de déploiement
      continue ; de la mise en place, au passage en production, d'une plateforme de documentation
      technique ou de solutions de recherches et de traitements documentaires spécifiques.
    }

  \cventry
    {\textbf{C++}, \textbf{docker}, \textbf{python}}
    {Ingénieur informatique}
    {WiseBim}
    {Travail à distance}
    {Janvier 2018 à Juillet 2018}{
      Mise en place d'une architecture permettant la génération automatique de
      maquettes numériques 3D (BIM) à partir de plans d'architecte en 2D de
      manière fiable et sécurisée à travers une interface web.
    }

  \cventry
    {\textbf{big data}, \textbf{python}, \textbf{C++}}
    {Ingénieur-Chercheur}
    {CEA/INES}
    {Chambéry}
    {Juin 2016 à Décembre 2017}{
      Étude et mise en place d'une grappe de serveurs et de son interface web
      pour stocker, présenter et analyser des flux importants de données
      disparates en provenance de capteurs.
    }

  \cventry
    {\textbf{ruby}, \\ \textbf{XML}, \textbf{DO178}}
    {Stagiaire ingénieur}
    {Thales Avionics SAS}
    {Valence}
    {6 mois}{
      Étude des besoins, réalisation et qualification selon la norme DO330
      d'un logiciel calculant des métriques de qualité concernant le code et
      les artefacts de la norme DO178 pour les logiciels critiques de
      l'avionique.
    }

%  \cventry
%    {\textbf{python}, \textbf{SQL}, \textbf{web}}
%    {Chef de projet et développeur}
%    {A2IA, INSA}
%    {Rouen}
%    {2 semestres}{
%      Gestion de projet et réalisation d'un outil de gestion de gros volumes de
%      métadonnées d'images pour la reconnaissance de caractères. Durant 25
%      heures par semaine en groupe de 15 en relation avec une sous traitance
%      interne dans le cadre de la formation INSA, et dans le respect de la
%      norme ISO9001:2008.
%    }

%  \cventry
%    {\textbf{python}, \textbf{SOAP}, \textbf{Qualité}}
%    {Développeur}
%    {Dynamease, INSA}
%    {Rouen}
%    {1 semestre}{
%      Réalisation de module pour openERP et d'un outil de gestion d'appel
%      téléphonique. Durant 25 heures par semaine en groupe de 7 dans le cadre
%      de la formation INSA et dans le respect de la norme ISO9001:2008.
%    }

  \cventry
    {\textbf{python}, \textbf{web}}
    {Stage de spécialité}
    {Bull}
    {Grenoble}
    {3 mois}{
      Réalisation d'un outil web de Quotation permettant de faciliter le choix
      des pièces lors de l'achat des composants d'un supercalculateur.
    }

%  \cventry
%    {\textbf{monde du travail}}
%    {Stage ouvrier}
%    {Autajon}
%    {Montélimar}
%    {9 semaines}{
%      Remise à jour et réorganisation des stocks d'outillage de l'atelier
%      découpe.
%    }

%  \cventry
%    {\textbf{monde du travail}}
%    {Agent d'entretien en contrat saisonnier}
%    {ASF, Autajon}
%    {Montélimar}
%    {2 étés}{
%      Nettoyage de toilettes d'autoroutes, nettoyage en hauteur à l’aide d'une
%      nacelle.
%    }

%  \cventry
%    {\textbf{HTML} \textbf{CSS} \textbf{javascript}}
%    {}
%    {Collège Europa}
%    {Montélimar}
%    {8 mois}{
%      Création et maintenance de sites en HTML dont l’intranet et le site
%      internet du collège Europa
%    }

%  \cventry
%    {\textbf{monde du travail} \textbf{Rédaction}}
%    {Stage de troisième}
%    {La tribune}
%    {Montélimar}
%    {Une semaine}{
%      Rédaction d'article, mise en page, critique de livre, « reportage » au
%      tournoi UNSS départemental de handball.
%    }

  \section{EXPÉRIENCE PERSONNELLE}

    \cventry
    {\textbf{Pédagogie}}
    {Encadrant}
    {Mixteen}
    {Lyon}
    {Mai 2019 à ce jour}{
      Encadrement d'enfants lors d'ateliers permettant de découvrir l’informatique de façon ludique avec
	  Scratch, Thymio, Pygame et Micro$:$bit en fonction de leur âge.
    }

    \cventry
    {\textbf{Gestion}, \textbf{Freelance}}
    {Mainteneur}
    {Open-source}
    {Github}
    {Mars 2017 à ce jour}{
      Développement et gestion (Pull-requests, publications sur Pypi et offres de freelance)
	  autour d'un module d'évaluation de la force de mots de passe, et d'une application web permettant de réaliser des
	  sondages.
    }

    \cventry
    {\textbf{python}, \textbf{Qualité}}
    {Membre}
    {Python Code Quality Authority}
    {Github}
    {Novembre 2016 à ce jour}{
      Contribution à l'évolution du logiciel \textbf{pylint}, un outil d'analyse statique de la qualité du code
      en python.
    }

    \cventry
    {\textbf{Gestion}, \textbf{Vidéo}}
    {Factotum}
    {Insa}
    {Rouen}
    {Février 2010 à Juin 2014}{
      En groupe de 3 à 5, scénarisation, réalisation et promotion d'une comédie
      féministe, d'un film d'action, d'un film de suspense et d'un western
      moderne dans le cadre de la Section Image Étude de l'INSA.
    }

%  \cvline
%    {\textbf{python}, \textbf{web}}
%    {
%      Reprise et amélioration du site de la mission qualité de l'INSA
%      (permettant de réaliser des statistiques sur la satisfaction des
%      personnels et élèves chaque semestre).
%    }{}{}

%  \cvline
%    {\textbf{python}, \textbf{web}}
%    {
%      Reconception de l'architecture et ajout de fonctionalité sur un logiciel
%      libre de gestion de bar (POSSUM).
%    }{}{}

%  \cvline
%    {\textbf{éléctronique}, \textbf{arduino}}
%    {
%      Conception et réalisation d'un cube en bois doté d'un clavier et de LED
%      tricolore contrôlée par une Arduino et permettant de jouer au morpion.
%    }{}{}


    \cventry
    {\textbf{C}, \textbf{\LaTeX{}}}
    {Développeur}
    {Insa}
    {Rouen}
    {Septembre 2011 à Juin 2012}{
      Conception et réalisation de l'intelligence artificielle d'un logiciel
      d'othello, et d'un programme arbitre pour tournoi d'othello entre
      logiciels.
    }

%  \cvline
%    {\textbf{C++}, \textbf{matlab}}
%    {
%      Conception et réalisation d'un mod de profilage de joueur sur le logiciel
%      libre Teeworld.
%    }{}{}

%  \cvline
%    {\textbf{java}, \textbf{IHM}}
%    {
%      Conception et réalisation d'un logiciel de musique assistée par
%      ordinateur permettant de composer des morceaux intuitivement. (Miraaz).
%    }{}{}

%  \cvline
%    {\textbf{java}, \textbf{IHM}}
%    {
%      Conception et réalisation d'une interface graphique 2D en Java pour un
%      mini jeu de rôle doté d'une scénarisation automatique et adaptative.
%    }{}{}

  \section{LANGUES}

  \cvlanguage
    {Anglais}
    {Experimenté (C2) (TOEIC 2013: 920/990)}
    {Traducteur bénévole pour Ubuntu, Jenkins, Leningrad, FlossManual et mes projets open-source}

  \cvlanguage
    {Espagnol}
    {Niveau Intermédiaire}
    {}{}{}{}

  \section{INFORMATIQUE}

  \cvcomputer
    {\textbf{Langages}}
    {Python, Bash, C/C++, SQL, Ruby}
    {\textbf{Outils}}
    {Git, Docker, Sphinx, Plantuml}

  \cvcomputer
    {\textbf{Frameworks}}
    {Django, Bootstrap (JQuery)}
    {\textbf{Logiciels}}
    {Gimp, Inkscape, Audacity, Suite Adobe}

  \cvcomputer
    {\textbf{DevOps}}
    {Déploiement continu, Apache, Kubernetes}
    {\textbf{Bureautique}}
    {\LaTeX{}, C2I niveau 2mi (LibreOffice, Office)}

\end{document}
